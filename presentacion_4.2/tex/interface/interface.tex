\section{Interface}
\begin{frame}{Interface}
\begin{itemize}{}
\item Son  superclases en las cuales la totalidad de sus m\'etodos se definen como  m\'etodos abstractos, que deben ser implementados en las subclases en las cuales la interface es implementada.
\item La interface define y conoce que métodos tiene, pero no sabe como es su implementaci\'n.
\item Por ejemplo la Interfaces Animal define el m\'etodo eat() sin conocer la l\'ogica
 de como cada tipo de animal se alimenta, luego todas las clases de animal que implementen
 \'esta interface definiran de distinta forma el m\'etodo eat() dependiendo de la naturaleza de cada animal.
\item En una interface en Java sus m\'etodos siempre  ser\'an p\'ublicos y abstractos.
\end{itemize}
\end{frame}

\begin{frame}{Ejemplo.}
	\begin{block}{Clase Main}
\lstinputlisting[language=Java,caption={},numbers=none]{resources/interface/IntegranteSeleccionFutbol.java}
\end{block}
\end{frame}

\begin{frame}{Ejemplo.}
	\begin{block}{Clase Main}
\lstinputlisting[language=Java,caption={},numbers=none]{resources/interface/Futbolista.java}
\end{block}
\end{frame}

\begin{frame}{Ejemplo.}
	\begin{block}{Clase Main}
\lstinputlisting[language=Java,caption={},numbers=none]{resources/interface/Entrenador.java}
\end{block}
\end{frame}
